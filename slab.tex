\documentclass{article}%
\usepackage[T1]{fontenc}%
\usepackage[utf8]{inputenc}%
\usepackage{lmodern}%
\usepackage{textcomp}%
\usepackage{lastpage}%
\usepackage{geometry}%
\geometry{tmargin=1cm,lmargin=1cm}%
%
%
%
\begin{document}%
\normalsize%
\section*{Design of slab}%
\label{sec:Designofslab}%
\subsection*{Trial depth:}%
\label{subsec:Trialdepth}%

%
As a starting point the depth of the slab will be set as 125 mm.%
\subsection*{General AS3600:2018 Cl 9.4.1}%
\label{subsec:GeneralAS36002018Cl9.4.1}%

%
The deflection of a slab shall be determined in accordance with Clause 9.4.2 or Clause 9.4.3.\newline%
\newline%
Alternatively, for reinforced slabs, the effective span{-}to{-}depth ratio of the slab shall conform with Clause 9.4.4.\newline%
\newline%
For a slab containing steel fibers in addition to conventional reinforcement or tendons, the deflection shall be determined in accordace with Clause 16.4.7.3.\newline%
%
\newline%
 Live load is greater than dead load, therefore Clause 9.4.4 deemed to comply deflection check is not applicable, use Clause 9.4.3.%
\subsection*{Slab deflection by simplified calculation  AS3600:2018 Cl 9.4.3}%
\label{subsec:SlabdeflectionbysimplifiedcalculationAS36002018Cl9.4.3}%

%
The deflection of a slab subject to uniformly distributed loads shall be calcualted in accordance with Clause 8.5.3 on the basis of an equivalent beam taken as follows:\newline%
\newline%
(a)	For a one{-}way slab, a prismatic beam of unit width.\newline%
\newline%
(b)	For a rectangular slab supported on four sides, a prismatic beam of unit width through the centre of the slab, spanning in the short direction %
$L_{x}$, with the same conditions of continuity as the slab in that direction and with the load distributedso that the proportion of load carried by the beam is given by:%
\newline%
\newline%
%
$\hspace*{10mm} L_{y}^{4} / (\alpha L_{x}^{4} + L_{y}^{4})$%
\newline%
\newline%
where %
 $\alpha$ is given in Table 9.4.3 for the appropriate slab-edge condition.%
\newline%
\newline%
(c)	 For a two{-}way flat slab having multiple spans (for deflections on the column lines or midway between the supports), the column strips of the idealised frame described in Clause 6.9.%
\subsection*{Table 9.4.3}%
\label{subsec:Table9.4.3}%
\begin{tabular}{l l | l }%
\hline%
&Edge condition&Coefficient ($\alpha$)\\%
\hline%
1&Four edges continuous&1\\%
2&One short edge discontinuous&0.5\\%
3&One long edge discontinuous&2\\%
4&Two short edges discontinuous&0.2\\%
5&Two long edges discontinuous&5\\%
6&Two adjacent edges discontinuous&1\\%
7&Three edges discontinuous (one long edge continuous)&0.4\\%
8&Three edges discontinuous (one short edge continuous)&2.5\\%
9&Four edges discontinuous&1\\%
\cline{1%
-%
3}%
\end{tabular}

%
\subsection*{Calculations}%
\label{subsec:Calculations}%

%
Lx = 3000mm\newline%
\newline%
Ly = 3000mm%
\newline%
\newline%
G = 5.5 KPa\newline%
Q = 4 KPa\newline%
\newline%
%
$(1.0 + k_{cs})G + (\psi_{s} + k_{cs}\psi_{l})Q = $12.6 KN/m%
\newline%
\newline%
%
For case 1%
\newline%
\newline%
%
$\alpha = 1.0%
\newline%
\newline%
%
L_{y}^{4} / (\alpha L_{x}^{4} + L_{y}^{4}) = $0.5%
\newline%
\newline%
 Load on unit width beam = 4.15KN/m%
\newline%
\newline%
%
For case 2%
\newline%
\newline%
%
$\alpha = 0.5%
\newline%
\newline%
%
L_{y}^{4} / (\alpha L_{x}^{4} + L_{y}^{4}) = $0.67%
\newline%
\newline%
 Load on unit width beam = 5.53KN/m%
\newline%
\newline%
%
For case 3%
\newline%
\newline%
%
$\alpha = 2.0%
\newline%
\newline%
%
L_{y}^{4} / (\alpha L_{x}^{4} + L_{y}^{4}) = $0.33%
\newline%
\newline%
 Load on unit width beam = 2.77KN/m%
\newline%
\newline%
%
For case 4%
\newline%
\newline%
%
$\alpha = 0.2%
\newline%
\newline%
%
L_{y}^{4} / (\alpha L_{x}^{4} + L_{y}^{4}) = $0.83%
\newline%
\newline%
 Load on unit width beam = 6.92KN/m%
\newline%
\newline%
%
For case 5%
\newline%
\newline%
%
$\alpha = 5.0%
\newline%
\newline%
%
L_{y}^{4} / (\alpha L_{x}^{4} + L_{y}^{4}) = $0.17%
\newline%
\newline%
 Load on unit width beam = 1.38KN/m%
\newline%
\newline%
%
For case 6%
\newline%
\newline%
%
$\alpha = 1.0%
\newline%
\newline%
%
L_{y}^{4} / (\alpha L_{x}^{4} + L_{y}^{4}) = $0.5%
\newline%
\newline%
 Load on unit width beam = 4.15KN/m%
\newline%
\newline%
%
For case 7%
\newline%
\newline%
%
$\alpha = 0.4%
\newline%
\newline%
%
L_{y}^{4} / (\alpha L_{x}^{4} + L_{y}^{4}) = $0.71%
\newline%
\newline%
 Load on unit width beam = 5.93KN/m%
\newline%
\newline%
%
For case 8%
\newline%
\newline%
%
$\alpha = 2.5%
\newline%
\newline%
%
L_{y}^{4} / (\alpha L_{x}^{4} + L_{y}^{4}) = $0.29%
\newline%
\newline%
 Load on unit width beam = 2.37KN/m%
\newline%
\newline%
%
For case 9%
\newline%
\newline%
%
$\alpha = 1.0%
\newline%
\newline%
%
L_{y}^{4} / (\alpha L_{x}^{4} + L_{y}^{4}) = $0.5%
\newline%
\newline%
 Load on unit width beam = 4.15KN/m%
\subsection*{Beam deflection by simplified calculation AS3600:2018 Cl 8.5.3}%
\label{subsec:BeamdeflectionbysimplifiedcalculationAS36002018Cl8.5.3}%

%
\subsection*{Short{-}term deflection AS3600:2018 Cl 8.5.3.1}%
\label{subsec:Short{-}termdeflectionAS36002018Cl8.5.3.1}%

%
The short{-}term deflection due to external loads and prestressing, which occur immediately on their application, shall be calculated using the value of %
$E_{cj}$ determined in accordance with Clause 3.1.2 and the value of the effective second moment of area of the member ($I_{ef}$).%
 This value of $I_{ef}$ shall be determined by rational calculation. Alternatively, $I_{ef}$ may be determined at the nominated cross-sections as follows:%
\newline%
\newline%
(a) For a simply suppported span, the value at midspan.\newline%
\newline%
(b)	In a continuous beam:\newline%
\newline%
		(i)	for an interior span, half the midspanvalue plus one quarter of each support value; or\newline%
\newline%
	(ii)	for an end span, half the midspan value plus half the value at the continuous support.%
\newline%
\newline%
(c)	For a cantilever, the value at the support.\newline%
\newline%
For the purposes of the above determinations, the value of %
$I_{ef}$ at each of the cross-sections nominated in Items (a) to (c) above is given by:%
\newline%
\newline%
%
$I_{ef} = \frac{I_{cr}}{1-\left(1-\frac{I_{cr}}{I}\right)\left(\frac{M_{cr.t}}{M_{s}^{*}}\right)^{2}} \leq I_{ef.max}%
\newline%
\newline%
where%
\newline%
\newline%
%
$I_{ef.max}$ = maximum effective second moment of area and is taken as $I$, for reinforced sections when\hspace{30mm} $p = A_{st}/bd \geq 0.005$ for prestressed sections%
\newline%
\newline%
%
$\hspace*{11mm} = $ 0.6$I$, for reinforced sections when $p = A_{st}/bd < 0.005$%
\newline%
\newline%
%
$\hspace*{15mm}b \hspace{5mm}= \hspace{5m}$width of the cross-section at the compression face%
\newline%
\newline%
%
$M_{s}^{*} = $ maximum bending moment at the section, based on the short-term serviceability load or the construction load%
\newline%
\newline%
%
$M_{cr.t} = Z(f'_{ct.f} - \sigma_{cs} + P / A_{g}) + Pe \geq 0$%
\newline%
\newline%
%
$\hspace*{15mm}Z = $ section modulus of the uncracked section, referred to the extreme fibre at which cracking occurs%
\newline%
\newline%
%
$\hspace*{15mm}f'_{ct.f} = $ characteristic flexural tensile strength of concrete%
\newline%
\newline%
%
$\hspace*{15mm}\sigma_{cs} = $ maximum shrinkage-induced tensile stress on the uncracked section at the extreme fibre at which cracking occurs. In the absence of more refined calculation, the value of $\sigma_{cs}$ that accounts for the restraint provided by the steel reinforcement may be taken as:%
\newline%
\newline%
%
$\hspace*{15mm} = \frac{2.5p_{w} - 0.8p_{cw}}{1 + 50P_{w}}E_{s}\varepsilon_{cs}%
\newline%
\newline%
%
\subsection*{Long{-}term deflection AS3600:2018  Cl 8.5.3.2}%
\label{subsec:Long{-}termdeflectionAS36002018Cl8.5.3.2}%

%
For reinforced and prestressed beams, that part of the deflection that occurs after the short term deflection shall be calcualted as the sum of:%
\newline%
\newline%
%
(a)$\hspace{10mm}$the shrinkage component of the long term deflection, determined from the design shrinkage strain of concrete ($\varepsilon_{cs}$) (see Clause 3.1.7) and the principles of mechanics; and%
\newline%
\newline%
%
(b)$\hspace{10mm}$the additional long-term creep deflections, determined from the design creep coefficient of concrete($\varphi_{cc}$) (see Clause 3.1.8) and the principles of mechanics.%
\newline%
\newline%
%
In the absence of more accurate calculations, the additional long-term deflection of a reinforced beam due to creep and shrinkage may be estimated by multiplying the short term-term deflection caused by the sustained loads (obtained using the final long-term shrinkage strain in the estimate of $M_{cr.t}$) by a multiplier, $K_{cs}$, given by:%
\newline%
\newline%
%
$k_{cs} = [2 - 1.2(A_{sc} / A_{st})] \geq 0.8$%
\newline%
\newline%
%
where $A_{sc}$ is the area of steel in the compressive zone of the cracked section between the neutral axis at service loads and the extreme concrete compressive fiber and $A_{sc}/A_{st}$ is taken at midspan, for a simply supported or continuous beam and at the support, for a cantilever beam.%
Long term deflection:%
\newline%
\newline%
%
$\Delta_{total} = \Delta_{s} + k_{cs}\Delta_{s.sus}%
\newline%
\newline%
%
where%
\newline%
\newline%
%
$\Delta{s} = \frac{L_{ox}^{2}}{96E_{c}I_{ef.av}}(M_{L.s} + 10M_{M.s} + M_{R.s})%
\newline%
\newline%
%
$\Delta_{s.sus} = \frac{w_{long}}{w_{short}}\Delta_{s}%
\subsection*{Calculations}%
\label{subsec:Calculations}%

%
Cases to check: {[}1, 2, 3, 4, 5, 6, 7, 8, 9{]}%
\newline%
\newline%
Try SL92 mesh top and bottom\newline%
\newline%
%
Assume reinforcement is uniform across slab.\newline%
\newline%
%
\newline%
\newline%
The value of dn is14.931707168572508\newline%
\newline%
%
\begin{tabular}{llll}%
\textbf{Slab Properties}&&&\\%
\hline%
Concrete cover&=&70& mm\\%
&&&\\%
Total slab depth&=&125& mm\\%
&&&\\%
Equivalent area of steel $A_{st}$&=&580& $mm^{2}$\\%
&&&\\%
Breadth, b &=&1000& mm\\%
&&&\\%
Depth to centroid of tension reinforcement, d&=&66& mm\\%
&&&\\%
Depth to the neutral axis, c&=&15& mm\\%
&&&\\%
Cracked second moment of area, $I_{cr}$&=&12551419& $mm^{4}\\%
&&&\\%
Gross second moment of area, $I_{g}$&=&162760417& $mm^{4}\\%
&&&\\%
Concrete flexural strength, $f'_{ct.f}&=&3.0& MPa\\%
&&&\\%
Concrete modulus of elasticity, $E_{c}$&=&26700.0& MPa\\%
&&&\\%
Steel modulus of elasticity, $E_{s}$&=&200000& MPa\\%
\end{tabular}%
\subsection*{For case 1}%
\label{subsec:Forcase1}%

%
\newline%
\newline%
%
\textbf{At the left support:}%
\newline%
\newline%
%
$M_{ct.t}$ = 2 KNm%
\newline%
%
$M_{s,L}^{*}$ = 4 KNm%
\newline%
%
$I_{ef,L}$ = 18407894 $mm^{4}$%
\newline%
\newline%
%
\textbf{At the midpoint support:}%
\newline%
\newline%
%
$M_{ct.t}$ = 2 KNm%
\newline%
%
$M_{s,M}^{*}$ = 1 KNm%
\newline%
%
$I_{ef,M}$ = 162760417 $mm^{4}$%
\newline%
\newline%
%
\textbf{At the right support:}%
\newline%
\newline%
%
$M_{ct.t}$ = 2 KNm%
\newline%
%
$M_{s,R}^{*}$ = 4 KNm%
\newline%
%
$I_{ef,R}$ = 18407894 $mm^{4}$%
\newline%
\newline%
%
Span Lx is continuous, therefore $I_{ef}$ = half the midspan value plus one quarter of each support%
\newline%
\newline%
%
$I_{ef} = 0.5I_{ef,M} + 0.25I_{ef,L} + 0.25I_{ef,R} = $90584156 $ mm^{4}$%
\newline%
\newline%
%
$\Delta_{s} =$0.072 mm%
\newline%
\newline%
%
$w_{long} = G + \psi_{l}Q = $7 KN/m%
\newline%
%
$w_{short} = G + \psi_{s}Q = $ 8 KN/m%
\newline%
\newline%
%
$\Delta_{s.sus}$ = 0.03 mm%
\newline%
\newline%
%
$\Delta_{total}$ = 0.13 mm%
\subsection*{For case 2}%
\label{subsec:Forcase2}%

%
\newline%
\newline%
%
\textbf{At the left support:}%
\newline%
\newline%
%
$M_{ct.t}$ = 2 KNm%
\newline%
%
$M_{s,L}^{*}$ = 5 KNm%
\newline%
%
$I_{ef,L}$ = 15287210 $mm^{4}$%
\newline%
\newline%
%
\textbf{At the midpoint support:}%
\newline%
\newline%
%
$M_{ct.t}$ = 2 KNm%
\newline%
%
$M_{s,M}^{*}$ = 1 KNm%
\newline%
%
$I_{ef,M}$ = 162760417 $mm^{4}$%
\newline%
\newline%
%
\textbf{At the right support:}%
\newline%
\newline%
%
$M_{ct.t}$ = 2 KNm%
\newline%
%
$M_{s,R}^{*}$ = 5 KNm%
\newline%
%
$I_{ef,R}$ = 15287210 $mm^{4}$%
\newline%
\newline%
%
Span Lx is continuous, therefore $I_{ef}$ = half the midspan value plus one quarter of each support%
\newline%
\newline%
%
$I_{ef} = 0.5I_{ef,M} + 0.25I_{ef,L} + 0.25I_{ef,R} = $89023814 $ mm^{4}$%
\newline%
\newline%
%
$\Delta_{s} =$0.098 mm%
\newline%
\newline%
%
$w_{long} = G + \psi_{l}Q = $7 KN/m%
\newline%
%
$w_{short} = G + \psi_{s}Q = $ 8 KN/m%
\newline%
\newline%
%
$\Delta_{s.sus}$ = 0.06 mm%
\newline%
\newline%
%
$\Delta_{total}$ = 0.21 mm%
\subsection*{For case 3}%
\label{subsec:Forcase3}%

%
\newline%
\newline%
%
\textbf{At the left support:}%
\newline%
\newline%
%
$M_{ct.t}$ = 2 KNm%
\newline%
%
$M_{s,L}^{*}$ = 2 KNm%
\newline%
%
$I_{ef,L}$ = 44169936 $mm^{4}$%
\newline%
\newline%
%
\textbf{At the midpoint support:}%
\newline%
\newline%
%
$M_{ct.t}$ = 2 KNm%
\newline%
%
$M_{s,M}^{*}$ = 1 KNm%
\newline%
%
$I_{ef,M}$ = 162760417 $mm^{4}$%
\newline%
\newline%
%
\textbf{At the right support:}%
\newline%
\newline%
%
$M_{ct.t}$ = 2 KNm%
\newline%
%
$M_{s,R}^{*}$ = 1 KNm%
\newline%
%
$I_{ef,R}$ = 162760417 $mm^{4}$%
\newline%
\newline%
%
Span Lx is an edge span, therefore $I_{ef}$ = half the midspan value plus half the value at the continuous support.%
\newline%
\newline%
%
$I_{ef} = 0.5I_{ef,M} + 0.5I_{ef,L} = $103465176 $ mm^{4}$%
\newline%
\newline%
%
$\Delta_{s} =$0.392 mm%
\newline%
\newline%
%
$w_{long} = G + \psi_{l}Q = $7 KN/m%
\newline%
%
$w_{short} = G + \psi_{s}Q = $ 8 KN/m%
\newline%
\newline%
%
$\Delta_{s.sus}$ = 0.11 mm%
\newline%
\newline%
%
$\Delta_{total}$ = 0.62 mm%
\subsection*{For case 4}%
\label{subsec:Forcase4}%

%
\newline%
\newline%
%
\textbf{At the left support:}%
\newline%
\newline%
%
$M_{ct.t}$ = 2 KNm%
\newline%
%
$M_{s,L}^{*}$ = 6 KNm%
\newline%
%
$I_{ef,L}$ = 14174932 $mm^{4}$%
\newline%
\newline%
%
\textbf{At the midpoint support:}%
\newline%
\newline%
%
$M_{ct.t}$ = 2 KNm%
\newline%
%
$M_{s,M}^{*}$ = 2 KNm%
\newline%
%
$I_{ef,M}$ = 162760417 $mm^{4}$%
\newline%
\newline%
%
\textbf{At the right support:}%
\newline%
\newline%
%
$M_{ct.t}$ = 2 KNm%
\newline%
%
$M_{s,R}^{*}$ = 6 KNm%
\newline%
%
$I_{ef,R}$ = 14174932 $mm^{4}$%
\newline%
\newline%
%
Span Lx is continuous, therefore $I_{ef}$ = half the midspan value plus one quarter of each support%
\newline%
\newline%
%
$I_{ef} = 0.5I_{ef,M} + 0.25I_{ef,L} + 0.25I_{ef,R} = $88467674 $ mm^{4}$%
\newline%
\newline%
%
$\Delta_{s} =$0.124 mm%
\newline%
\newline%
%
$w_{long} = G + \psi_{l}Q = $7 KN/m%
\newline%
%
$w_{short} = G + \psi_{s}Q = $ 8 KN/m%
\newline%
\newline%
%
$\Delta_{s.sus}$ = 0.09 mm%
\newline%
\newline%
%
$\Delta_{total}$ = 0.3 mm%
\subsection*{For case 5}%
\label{subsec:Forcase5}%

%
\newline%
\newline%
%
\textbf{At the left support:}%
\newline%
\newline%
%
$M_{ct.t}$ = 2 KNm%
\newline%
%
$M_{s,L}^{*}$ = 0 KNm%
\newline%
%
$I_{ef,L}$ = 162760417 $mm^{4}$%
\newline%
\newline%
%
\textbf{At the midpoint support:}%
\newline%
\newline%
%
$M_{ct.t}$ = 2 KNm%
\newline%
%
$M_{s,M}^{*}$ = 2 KNm%
\newline%
%
$I_{ef,M}$ = 162760417 $mm^{4}$%
\newline%
\newline%
%
\textbf{At the right support:}%
\newline%
\newline%
%
$M_{ct.t}$ = 2 KNm%
\newline%
%
$M_{s,R}^{*}$ = 0 KNm%
\newline%
%
$I_{ef,R}$ = 162760417 $mm^{4}$%
\newline%
\newline%
%
Span Lx is simply supported, therefore $I_{ef}$ = the value at midspan%
\newline%
\newline%
%
$I_{ef} = I_{ef,M} = $162760417 $ mm^{4}$%
\newline%
\newline%
%
$\Delta_{s} =$0.319 mm%
\newline%
\newline%
%
$w_{long} = G + \psi_{l}Q = $7 KN/m%
\newline%
%
$w_{short} = G + \psi_{s}Q = $ 8 KN/m%
\newline%
\newline%
%
$\Delta_{s.sus}$ = 0.05 mm%
\newline%
\newline%
%
$\Delta_{total}$ = 0.41 mm%
\subsection*{For case 6}%
\label{subsec:Forcase6}%

%
\newline%
\newline%
%
\textbf{At the left support:}%
\newline%
\newline%
%
$M_{ct.t}$ = 2 KNm%
\newline%
%
$M_{s,L}^{*}$ = 4 KNm%
\newline%
%
$I_{ef,L}$ = 18407894 $mm^{4}$%
\newline%
\newline%
%
\textbf{At the midpoint support:}%
\newline%
\newline%
%
$M_{ct.t}$ = 2 KNm%
\newline%
%
$M_{s,M}^{*}$ = 2 KNm%
\newline%
%
$I_{ef,M}$ = 122989388 $mm^{4}$%
\newline%
\newline%
%
\textbf{At the right support:}%
\newline%
\newline%
%
$M_{ct.t}$ = 2 KNm%
\newline%
%
$M_{s,R}^{*}$ = 1 KNm%
\newline%
%
$I_{ef,R}$ = 162760417 $mm^{4}$%
\newline%
\newline%
%
Span Lx is an edge span, therefore $I_{ef}$ = half the midspan value plus half the value at the continuous support.%
\newline%
\newline%
%
$I_{ef} = 0.5I_{ef,M} + 0.5I_{ef,L} = $70698641 $ mm^{4}$%
\newline%
\newline%
%
$\Delta_{s} =$0.861 mm%
\newline%
\newline%
%
$w_{long} = G + \psi_{l}Q = $7 KN/m%
\newline%
%
$w_{short} = G + \psi_{s}Q = $ 8 KN/m%
\newline%
\newline%
%
$\Delta_{s.sus}$ = 0.37 mm%
\newline%
\newline%
%
$\Delta_{total}$ = 1.6 mm%
\subsection*{For case 7}%
\label{subsec:Forcase7}%

%
\newline%
\newline%
%
\textbf{At the left support:}%
\newline%
\newline%
%
$M_{ct.t}$ = 2 KNm%
\newline%
%
$M_{s,L}^{*}$ = 5 KNm%
\newline%
%
$I_{ef,L}$ = 14869475 $mm^{4}$%
\newline%
\newline%
%
\textbf{At the midpoint support:}%
\newline%
\newline%
%
$M_{ct.t}$ = 2 KNm%
\newline%
%
$M_{s,M}^{*}$ = 3 KNm%
\newline%
%
$I_{ef,M}$ = 22413012 $mm^{4}$%
\newline%
\newline%
%
\textbf{At the right support:}%
\newline%
\newline%
%
$M_{ct.t}$ = 2 KNm%
\newline%
%
$M_{s,R}^{*}$ = 2 KNm%
\newline%
%
$I_{ef,R}$ = 162760417 $mm^{4}$%
\newline%
\newline%
%
Span Lx is an edge span, therefore $I_{ef}$ = half the midspan value plus half the value at the continuous support.%
\newline%
\newline%
%
$I_{ef} = 0.5I_{ef,M} + 0.5I_{ef,L} = $18641244 $ mm^{4}$%
\newline%
\newline%
%
$\Delta_{s} =$4.663 mm%
\newline%
\newline%
%
$w_{long} = G + \psi_{l}Q = $7 KN/m%
\newline%
%
$w_{short} = G + \psi_{s}Q = $ 8 KN/m%
\newline%
\newline%
%
$\Delta_{s.sus}$ = 2.85 mm%
\newline%
\newline%
%
$\Delta_{total}$ = 10.36 mm%
\subsection*{For case 8}%
\label{subsec:Forcase8}%

%
\newline%
\newline%
%
\textbf{At the left support:}%
\newline%
\newline%
%
$M_{ct.t}$ = 2 KNm%
\newline%
%
$M_{s,L}^{*}$ = 1 KNm%
\newline%
%
$I_{ef,L}$ = 162760417 $mm^{4}$%
\newline%
\newline%
%
\textbf{At the midpoint support:}%
\newline%
\newline%
%
$M_{ct.t}$ = 2 KNm%
\newline%
%
$M_{s,M}^{*}$ = 3 KNm%
\newline%
%
$I_{ef,M}$ = 33343692 $mm^{4}$%
\newline%
\newline%
%
\textbf{At the right support:}%
\newline%
\newline%
%
$M_{ct.t}$ = 2 KNm%
\newline%
%
$M_{s,R}^{*}$ = 1 KNm%
\newline%
%
$I_{ef,R}$ = 162760417 $mm^{4}$%
\newline%
\newline%
%
Span Lx is simply supported, therefore $I_{ef}$ = the value at midspan%
\newline%
\newline%
%
$I_{ef} = I_{ef,M} = $33343692 $ mm^{4}$%
\newline%
\newline%
%
$\Delta_{s} =$2.669 mm%
\newline%
\newline%
%
$w_{long} = G + \psi_{l}Q = $7 KN/m%
\newline%
%
$w_{short} = G + \psi_{s}Q = $ 8 KN/m%
\newline%
\newline%
%
$\Delta_{s.sus}$ = 0.65 mm%
\newline%
\newline%
%
$\Delta_{total}$ = 3.97 mm%
\subsection*{For case 9}%
\label{subsec:Forcase9}%

%
\newline%
\newline%
%
\textbf{At the left support:}%
\newline%
\newline%
%
$M_{ct.t}$ = 2 KNm%
\newline%
%
$M_{s,L}^{*}$ = 1 KNm%
\newline%
%
$I_{ef,L}$ = 162760417 $mm^{4}$%
\newline%
\newline%
%
\textbf{At the midpoint support:}%
\newline%
\newline%
%
$M_{ct.t}$ = 2 KNm%
\newline%
%
$M_{s,M}^{*}$ = 5 KNm%
\newline%
%
$I_{ef,M}$ = 15760514 $mm^{4}$%
\newline%
\newline%
%
\textbf{At the right support:}%
\newline%
\newline%
%
$M_{ct.t}$ = 2 KNm%
\newline%
%
$M_{s,R}^{*}$ = 1 KNm%
\newline%
%
$I_{ef,R}$ = 162760417 $mm^{4}$%
\newline%
\newline%
%
Span Lx is simply supported, therefore $I_{ef}$ = the value at midspan%
\newline%
\newline%
%
$I_{ef} = I_{ef,M} = $15760514 $ mm^{4}$%
\newline%
\newline%
%
$\Delta_{s} =$9.881 mm%
\newline%
\newline%
%
$w_{long} = G + \psi_{l}Q = $7 KN/m%
\newline%
%
$w_{short} = G + \psi_{s}Q = $ 8 KN/m%
\newline%
\newline%
%
$\Delta_{s.sus}$ = 4.23 mm%
\newline%
\newline%
%
$\Delta_{total}$ = 18.33 mm%
\newline%
\newline%
%
\begin{tabular}{llll}%
Summary of deflections&&&\\%
\hline%
1&0.13& mm&OK\\%
2&0.21& mm&OK\\%
3&0.62& mm&OK\\%
4&0.3& mm&OK\\%
5&0.41& mm&OK\\%
6&1.6& mm&OK\\%
7&10.36& mm&OK\\%
8&3.97& mm&OK\\%
9&18.33& mm&NG\\%
\end{tabular}%
\section*{Strength of Slabs in Bending AS3600:2018 Cl 9.1}%
\label{sec:StrengthofSlabsinBendingAS36002018Cl9.1}%
\subsection*{General AS3600:2018 Cl 9.1.1}%
\label{subsec:GeneralAS36002018Cl9.1.1}%

%
The strength of a slab in bending shall be determined in accordance with Clauses 8.1.1 to 8.1.8, except that for a two{-}way reinforced slabs, the minimum strength requirements of Clause 8.1.6.1 shall be determined to be satisfied by providing tensile reinforcement such that %
$A_{st}/bd$ is not less than the following in each direction:%
\newline%
\newline%
%
\begin{tabular}{lll}%
(a)&Slabs supported by columns at their corners&$\hspace*{10mm}0.24(D/d)^{2}f'_{ct.f}/f_{sy}$\\%
&&\\%
(b)&Slabs supported by beams or walls on four sides&$\hspace*{10mm}0.19(D/d)^{2}f'_{ct.f}/f_{sy}$\\%
\end{tabular}%
\newline%
\newline%
%
\subsection*{Strength of Beams in Bending AS3600:2018 Cl 8.1}%
\label{subsec:StrengthofBeamsinBendingAS36002018Cl8.1}%

%
\subsection*{Basis of strength calculations AS3600:2018 Cl 8.1.2}%
\label{subsec:BasisofstrengthcalculationsAS36002018Cl8.1.2}%

%
\subsection*{Rectangular stress block AS3600:2018 Cl 8.1.3}%
\label{subsec:RectangularstressblockAS36002018Cl8.1.3}%

%
\subsection*{Calculations}%
\label{subsec:Calculations}%

%
Determine bending capacity of section ($\phi M_{n}$):%
\newline%
\newline%
%
$\phi M_{n} = $11.4 KNm%
\newline%
\newline%
Determine bending moment demands:\newline%
\newline%
As reo is symmetrical top and bottom, %
$\phi M_{n}$ applies to positive and negative bending

%
\newline%
\newline%
%
$M_{x}^{*} = \beta_{x} F_{d} L_{x}^{2}$%
\newline%
\newline%
%
$M_{y}^{*} = \beta_{y} F_{d} L_{x}^{2}$%
\newline%
\newline%
where %
$F_{d}$ is the uniformly distributed design load per unit area factored for strength, and $\beta_{x}$ and $\beta_{y}$ are given in:%
\newline%
\newline%
%
\begin{tabular}{ll}%
(i)&Table 6.10.3.2(A) for slabs with Ductility Class N reinforcement as the main flexural reinforcement\\%
&\\%
(ii)&Table 6.10.3.2(B) for slabs with Ductility Class N reinforcement or Ductility Class L mesh as the main\\%
& flexural reinforcement, no moment redistribution can be accommodated at either the servicibility or strength limit states\\%
\end{tabular}%
\newline%
\newline%
%
The moments, so calcualted, shall apply over a central region of the slab equal to three quarters of $L_{x}$ and $L_{y}$ respectively. Outside of this region, the minimum requirement for strength shall be deemed to conform with the minimum strength requirement of Clause 9.1.1.%
\newline%
\newline%
%
(b)$\hspace{10mm}$The negative bending moments at a continuous edge shall be taken as:%
\newline%
\newline%
%
\begin{tabular}{lll}%
&(i)&1.33 times the midspan values in the direction considered, when they are taken from Table 6.10.3.2(A)\\%
&&\\%
&(ii)&$\alpha_{x}$ or $\alpha_{y}$ times the midspan values in the direction considered when they are taken from Table 6.10.3.2(B)\\%
\end{tabular}%
\newline%
\newline%
%
\subsection{Table 6.10.3.2(B)}%
\label{subsec:Table6.10.3.2(B)}%

%
\subsection{Bending moment coefficients for rectangular slabs supported on four sides (ductility class N or ductility class L reinforcement)}%
\label{subsec:Bendingmomentcoefficientsforrectangularslabssupportedonfoursides(ductilityclassNorductilityclassLreinforcement)}%

%
\begin{tabular}{|l|l|l|l|l|l|l|l|l|l|l|l|l}%
\multicolumn{3}{|c|}{Edge condition}&\multicolumn{9}{|c|}{Short span coefficients (\$\textbackslash{}beta\_\{x\}\$ and \$\textbackslash{}alpha\_\{x\}\$}&Long span coefficients ($eta_{y}$ and $lpha_{y}$) for all values of $L_{y}/L_{x}$\\%
\end{tabular}%
\subsection{Calculations}%
\label{subsec:Calculations}%

%
\textbf{For case 1.\newline%
\newline%
}%
$M_{x}^{*}$ = 2.4 KNm%
\newline%
\newline%
%
$M_{y}^{*}$ = 2.3 KNm%
\newline%
\newline%
%
$M_{x,negative}^{*}$ = 5.5 KNm%
\newline%
\newline%
%
$M_{y,negative}^{*}$ = 6.1 KNm%
\newline%
\newline%
%
\textbf{For case 2.\newline%
\newline%
}%
$M_{x}^{*}$ = 3.1 KNm%
\newline%
\newline%
%
$M_{y}^{*}$ = 2.7 KNm%
\newline%
\newline%
%
$M_{x,negative}^{*}$ = 6.7 KNm%
\newline%
\newline%
%
$M_{y,negative}^{*}$ = 6.2 KNm%
\newline%
\newline%
%
\textbf{For case 3.\newline%
\newline%
}%
$M_{x}^{*}$ = 2.7 KNm%
\newline%
\newline%
%
$M_{y}^{*}$ = 3.2 KNm%
\newline%
\newline%
%
$M_{x,negative}^{*}$ = 6.0 KNm%
\newline%
\newline%
%
$M_{y,negative}^{*}$ = 7.8 KNm%
\newline%
\newline%
%
\textbf{For case 4.\newline%
\newline%
}%
$M_{x}^{*}$ = 3.6 KNm%
\newline%
\newline%
%
$M_{y}^{*}$ = 2.7 KNm%
\newline%
\newline%
%
$M_{x,negative}^{*}$ = 7.6 KNm%
\newline%
\newline%
%
$M_{y,negative}^{*}$ = 0.0 KNm%
\newline%
\newline%
%
\textbf{For case 5.\newline%
\newline%
}%
$M_{x}^{*}$ = 2.7 KNm%
\newline%
\newline%
%
$M_{y}^{*}$ = 4.4 KNm%
\newline%
\newline%
%
$M_{x,negative}^{*}$ = 0.0 KNm%
\newline%
\newline%
%
$M_{y,negative}^{*}$ = 10.2 KNm%
\newline%
\newline%
%
\textbf{For case 6.\newline%
\newline%
}%
$M_{x}^{*}$ = 3.5 KNm%
\newline%
\newline%
%
$M_{y}^{*}$ = 3.9 KNm%
\newline%
\newline%
%
$M_{x,negative}^{*}$ = 7.5 KNm%
\newline%
\newline%
%
$M_{y,negative}^{*}$ = 8.2 KNm%
\newline%
\newline%
%
\textbf{For case 7.\newline%
\newline%
}%
$M_{x}^{*}$ = 4.4 KNm%
\newline%
\newline%
%
$M_{y}^{*}$ = 4.0 KNm%
\newline%
\newline%
%
$M_{x,negative}^{*}$ = 9.0 KNm%
\newline%
\newline%
%
$M_{y,negative}^{*}$ = 0.0 KNm%
\newline%
\newline%
%
\textbf{For case 8.\newline%
\newline%
}%
$M_{x}^{*}$ = 3.7 KNm%
\newline%
\newline%
%
$M_{y}^{*}$ = 5.2 KNm%
\newline%
\newline%
%
$M_{x,negative}^{*}$ = 0.0 KNm%
\newline%
\newline%
%
$M_{y,negative}^{*}$ = 11.1 KNm%
\newline%
\newline%
%
\textbf{For case 9.\newline%
\newline%
}%
$M_{x}^{*}$ = 5.0 KNm%
\newline%
\newline%
%
$M_{y}^{*}$ = 5.6 KNm%
\newline%
\newline%
%
$M_{x,negative}^{*}$ = 0.0 KNm%
\newline%
\newline%
%
$M_{y,negative}^{*}$ = 0.0 KNm%
\newline%
\newline%

%
\end{document}